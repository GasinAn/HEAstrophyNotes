\chapter{高能辐射机制简述}

\section{辐射传播相关物理量的定义}

$I_\nu(\theta,\phi)$: specific intensity (brightness, 亮度), 单位面积, 单位时间, 单位频率, 法线方向单位立体角穿过的能量,
\begin{equation}
    I_\nu(\theta,\phi)=\frac{d E}{d A\, d t\, d\nu \,d\Omega} .
\end{equation}
若 法线方向沿 $z$ 轴正方向, 且不要求法线方向穿过,
\begin{equation}
    I_\nu(\theta,\phi)=\frac{d E}{(d A\cos\theta) \,d t \,d\nu \,d\Omega} .
\end{equation}
$J_\nu$: 平均强度,
\begin{equation}
    J_\nu=\frac{1 }{4\pi}\int I_\nu(\theta,\phi)\,d\Omega .
\end{equation}
在自由空间,沿视线方向,辐射强度不变.

$F_\nu$:  monochromatic energy flux (单色能流量): 单位面积, 单位时间, 在单位频率间隔穿过的能量,
\begin{equation}
    F_\nu=\int I_\nu(\theta,\phi)\cos\theta \,d\Omega . 
\end{equation}
$\mathcal{F}$: energy fluence (能流): 单位面积穿过的能量, 
\begin{equation}
    \mathcal{F}=\int F_\nu\,dt \,d\nu .
\end{equation}

$u_\nu(\theta,\phi)$: radiative energy density 辐射能量密度,
\begin{equation}
    u_\nu(\theta,\phi)=\frac{d E}{(d A \,c\,d t )\, d\nu \,d\Omega},
\end{equation}
\begin{equation}
    u_\nu(\theta,\phi)=\frac{1}{c}I_\nu(\theta,\phi).
\end{equation}
\begin{equation}
    u_\nu=\frac{4\pi}{c}J_\nu.
\end{equation}

整理: specific intensity / brightness
\begin{align}
    dE_\nu(\Omega)=I_\nu(\Omega)\,dA_\perp(\Omega)\,dt\,d\Omega\,d\nu,
\end{align}
net flux
\begin{align}
    dE_\nu&=F_\nu\,dA\,dt\,d\nu\\
    &=\int_\Omega I_\nu(\Omega)\,dA_\perp(\Omega)\,dt\,d\Omega\,d\nu\\
    &=\int_\Omega I_\nu(\Omega)\,dA\cos\theta\,dt\,d\Omega\,d\nu\\
    &=\left(\int_\Omega I_\nu(\Omega)\cos\theta\,d\Omega\right)\,dA\,dt\,d\nu,
\end{align}
momentum flux
\begin{align}
    d{P_\perp}_\nu(\Omega)=d{p_\perp}_\nu(\Omega)\,dA\,dt\,d\nu,
\end{align}
\begin{align}
    d{P_\perp}_\nu(\Omega)&=dP_\nu(\Omega)\cos\theta\\
    &=(dE_\nu(\Omega)/c)\cos\theta\\
    &=(I_\nu(\Omega)/c)\cos^2\theta\,d\Omega\,dA\,dt\,d\nu,
\end{align}
energy density
\begin{align}
    dE_\nu(\Omega)&=u_\nu(\Omega)\,dV_\perp(\Omega)\,d\Omega\,d\nu\\
    &=u_\nu(\Omega)\,dA_\perp(\Omega)\,cdt\,d\Omega\,d\nu,
\end{align}
\begin{align}
    dE_\nu(\Omega)=I_\nu(\Omega)\,dA_\perp(\Omega)\,dt\,d\Omega\,d\nu,
\end{align}
\begin{align}
    J_\nu&=\frac{\int I_\nu(\Omega)\,d\Omega}{\int \,d\Omega}\\
    &=\int (I_\nu(\Omega)/4\pi)\,d\Omega.
\end{align}

\begin{equation}
    {I_\nu}_1(\Omega)\,{dA_\perp}_1(\Omega)\,dt\,{d\Omega}_1\,d\nu={I_\nu}_2(\Omega)\,{dA_\perp}_2(\Omega)\,dt\,{d\Omega}_2\,d\nu,
\end{equation}
\begin{equation}
    {d\Omega}_1={dA_\perp}_2(\Omega)/r^2,\quad
    {d\Omega}_2={dA_\perp}_1(\Omega)/r^2,
\end{equation}
\begin{equation}
    {I_\nu}_1(\Omega)={I_\nu}_2(\Omega).
\end{equation}

\section{辐射转移}

\begin{align}
    ddE_\nu(\Omega)&=\epsilon_\nu(\Omega)\,dV_\perp(\Omega)\,dt\,d\Omega\,d\nu\\
    &=\epsilon_\nu(\Omega)\,dA_\perp(\Omega)\,ds\,dt\,d\Omega\,d\nu\\
    &=dI_\nu(\Omega)\,dA_\perp(\Omega)\,dt\,d\Omega\,d\nu
\end{align}
\begin{align}
    dI_\nu(\Omega)=\epsilon_\nu(\Omega)\,ds,
\end{align}
\begin{align}
    dS_\perp(\Omega)&=\sigma_\nu\,dN_\perp(\Omega)\\
    &=\sigma_\nu n\,dV_\perp(\Omega)\\
    &=\sigma_\nu n\,dA_\perp(\Omega)\,ds,
\end{align}
\begin{align}
    ddE_\nu(\Omega)&=-I_\nu(\Omega)\,dS_\perp(\Omega)\,dt\,d\Omega\,d\nu\\
    &=-I_\nu(\Omega)\sigma_\nu n\,dA_\perp(\Omega)\,ds\,dt\,d\Omega\,d\nu\\
    &=-I_\nu(\Omega)\alpha_\nu\,dA_\perp(\Omega)\,ds\,dt\,d\Omega\,d\nu\\
    &=dI_\nu(\Omega)\,dA_\perp(\Omega)\,dt\,d\Omega\,d\nu
\end{align}
\begin{align}
    dI_\nu(\Omega)=-I_\nu(\Omega)\alpha_\nu\,ds,
\end{align}

\section{辐射机制}

\subsection{黑体辐射}

热平衡: 宏观无热量交换. 热辐射: 辐射源热平衡. 黑体辐射: 辐射源热平衡+辐射全被辐射源吸收 $\Rightarrow$ 辐射热平衡.

$\hbar\omega/kT\ll 1$, $\propto\omega^2$ $\hbar\omega/kT\gg 1$, $\propto e^{-\hbar\omega/kT}$. $\hbar\omega_\text{W}/kT\approx2.82$.

\subsection{加速带电粒子的电磁辐射}

\begin{equation}
    \vec{A}(t,\vec{r})=\frac{\mu_0}{4\pi}\int\frac{\vec{j}(t-|\vec{r}-\vec{r}'|,\vec{r}')}{|\vec{r}-\vec{r}'|}dV'
\end{equation}
$\vec{r}_e(t)$, $\vec{j}(t,\vec{r}')=q\delta(\vec{r}'-\vec{r}_e)\vec{v}_e$, $K=1-\frac{\vec{r}-\vec{r}_e}{|\vec{r}-\vec{r}_e|}\vec{v}_e$,
\begin{equation}
    \vec{A}(t,\vec{r})=\left.\left[\frac{q\vec{v}_e}{K|\vec{r}-\vec{r}_e|}\right]\right|_{t=t-|\vec{r}-\vec{r}_e|}
\end{equation}

\subsubsection{电场中单电子的辐射}

$I_\parallel/I_\perp\approx 1/\gamma^2$, $\omega\approx\gamma v/b$.

\paragraph{热韧致辐射}

$\tau\gg 1$, $\propto\omega^2$, $\tau\ll 1 \wedge \hbar\omega/kT\ll 1$, $\approx\text{const}$, $\hbar\omega/kT\gg 1$, $\propto e^{-\hbar\omega/kT}$.

\paragraph{非热韧致辐射}

\subsection{磁场中单电子的辐射}

$evB=\gamma m \omega^2 r=\gamma m \omega v$, $\omega_\text{g}=eB/\gamma m=\omega_\text{cycle}/\gamma$.

\paragraph{回旋辐射}

\paragraph{同步辐射}

周期 $2\pi/\omega_\text{g}$, 宽度 $\approx1/\gamma^3\omega_\text{g}$.

$\omega/\omega_\text{c}\ll 1$, $\propto\omega^{1/3}$, $\omega/\omega_\text{c}\gg 1$, $\propto e^{-\omega/\omega_\text{c}}$. $\omega_\text{c}=3\gamma^2eB/2m=(3/2)\gamma^3\omega_\text{g}$.

$\tau\gg 1$, $I(\omega)\propto\omega^{5/2}$, $\tau\ll 1$, $N(E)\propto E^{-p}$, $I(\omega)\propto \omega^{-(p-1)/2}$.

\subsection{光子散射}

\subsubsection{Thomson 散射}

   $m_ec^2 = \frac{e^2}{4\pi\epsilon_0 r_e}$
\begin{equation}
    \frac{d \sigma_\text{T}}{d \Omega}:=\frac{\frac{d E_\text{out}}{dt d\Omega}}{S_\text{in}}=\frac{\frac{d E_\text{out}}{dt d\Omega}}{\frac{d E_\text{in}}{dt dA}},\quad S_\text{in}\sigma_\text{T}=\frac{d E_\text{out}}{dt}
\end{equation}

\subsubsection{Compton 散射}

$\hbar\omega/m_ec^2\ll1$, $\sigma\approx\sigma_\text{T}$, $\hbar\omega/m_ec^2\gg1$, $\sigma\propto\omega^{-1}$.

\subsubsection{逆 Compton 散射}

$\omega\le 4\gamma^2\omega_0$. $\omega\ll 4\gamma^2\omega_0$, $\propto\omega$
