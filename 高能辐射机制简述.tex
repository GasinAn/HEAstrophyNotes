\chapter{高能辐射机制简述}

\section{辐射传播相关物理量的定义}

$I_\nu(\theta,\phi)$: specific intensity (brightness, 亮度), 单位面积, 单位时间, 单位频率, 法线方向单位立体角穿过的能量,
\begin{equation}
    I_\nu(\theta,\phi)=\frac{d E}{d A\, d t\, d\nu \,d\Omega} .
\end{equation}
若 法线方向沿 $z$ 轴正方向, 且不要求法线方向穿过,
\begin{equation}
    I_\nu(\theta,\phi)=\frac{d E}{(d A\cos\theta) \,d t \,d\nu \,d\Omega} .
\end{equation}
$J_\nu$: 平均强度,
\begin{equation}
    J_\nu=\frac{1 }{4\pi}\int I_\nu(\theta,\phi)\,d\Omega .
\end{equation}
在自由空间,沿视线方向,辐射强度不变.

$F_\nu$:  monochromatic energy flux (单色能流量): 单位面积, 单位时间, 在单位频率间隔穿过的能量,
\begin{equation}
    F_\nu=\int I_\nu(\theta,\phi)\cos\theta \,d\Omega . 
\end{equation}
$\mathcal{F}$: energy fluence (能流): 单位面积穿过的能量, 
\begin{equation}
    \mathcal{F}=\int F_\nu\,dt \,d\nu .
\end{equation}

$u_\nu(\theta,\phi)$: radiative energy density 辐射能量密度,
\begin{equation}
    u_\nu(\theta,\phi)=\frac{d E}{(d A \,c\,d t )\, d\nu \,d\Omega},
\end{equation}
\begin{equation}
    u_\nu(\theta,\phi)=\frac{1}{c}I_\nu(\theta,\phi).
\end{equation}
\begin{equation}
    u_\nu=\frac{4\pi}{c}J_\nu.
\end{equation}

整理: specific intensity / brightness
\begin{align}
    dE_\nu(\Omega)=I_\nu(\Omega)\,dA_\perp(\Omega)\,dt\,d\Omega\,d\nu,
\end{align}
net flux
\begin{align}
    dE_\nu&=F_\nu\,dA\,dt\,d\nu\\
    &=\int_\Omega I_\nu(\Omega)\,dA_\perp(\Omega)\,dt\,d\Omega\,d\nu\\
    &=\int_\Omega I_\nu(\Omega)\,dA\cos\theta\,dt\,d\Omega\,d\nu\\
    &=\left(\int_\Omega I_\nu(\Omega)\cos\theta\,d\Omega\right)\,dA\,dt\,d\nu,
\end{align}
momentum flux
\begin{align}
    d{P_\perp}_\nu(\Omega)=d{p_\perp}_\nu(\Omega)\,dA\,dt\,d\nu,
\end{align}
\begin{align}
    d{P_\perp}_\nu(\Omega)&=dP_\nu(\Omega)\cos\theta\\
    &=(dE_\nu(\Omega)/c)\cos\theta\\
    &=(I_\nu(\Omega)/c)\cos^2\theta\,d\Omega\,dA\,dt\,d\nu,
\end{align}
energy density
\begin{align}
    dE_\nu(\Omega)&=u_\nu(\Omega)\,dV_\perp(\Omega)\,d\Omega\,d\nu\\
    &=u_\nu(\Omega)\,dA_\perp(\Omega)\,cdt\,d\Omega\,d\nu,
\end{align}
\begin{align}
    dE_\nu(\Omega)=I_\nu(\Omega)\,dA_\perp(\Omega)\,dt\,d\Omega\,d\nu,
\end{align}
\begin{align}
    J_\nu&=\frac{\int I_\nu(\Omega)\,d\Omega}{\int \,d\Omega}\\
    &=\int (I_\nu(\Omega)/4\pi)\,d\Omega.
\end{align}

\begin{equation}
    {I_\nu}_1(\Omega)\,{dA_\perp}_1(\Omega)\,dt\,{d\Omega}_1\,d\nu={I_\nu}_2(\Omega)\,{dA_\perp}_2(\Omega)\,dt\,{d\Omega}_2\,d\nu,
\end{equation}
\begin{equation}
    {d\Omega}_1={dA_\perp}_2(\Omega)/r^2,\quad
    {d\Omega}_2={dA_\perp}_1(\Omega)/r^2,
\end{equation}
\begin{equation}
    {I_\nu}_1(\Omega)={I_\nu}_2(\Omega).
\end{equation}

\section{辐射转移}

\section{辐射机制}

\subsection{黑体辐射}

\subsection{加速带电粒子的电磁辐射}

\subsubsection{电场中单电子的辐射}

\paragraph{热韧致辐射}

\paragraph{非热韧致辐射}

\subsection{磁场中单电子的辐射}

\paragraph{回旋辐射}

\paragraph{同步辐射}

\subsection{光子散射}

\subsubsection{Thomson 散射}

\subsubsection{Compton 散射}

\subsubsection{逆 Compton 散射}
