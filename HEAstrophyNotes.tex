% 编译方式: xelatex*2
\documentclass{ctexbook}
\usepackage{amsfonts}
\usepackage{amsmath}
\usepackage{amssymb}
\usepackage{hyperref}
\usepackage{syntonly}
\usepackage{IEEEtrantools}
%\syntaxonly
\pagestyle{plain}
\makeatletter
\newcommand{\starttoc}{
    \chapter*{\contentsname}
    \@starttoc{toc}
}
\makeatother
%\renewcommand{\tableofcontents}{\twocolumn\starttoc\onecolumn}
\hypersetup{
    colorlinks,
    linkcolor=blue,
    filecolor=pink,
    urlcolor=cyan,
    citecolor=red,
}
\def\b{\boldsymbol}
\def\d{\mathrm{d}}
\makeatletter
\def\@begintheorem#1#2{\trivlist
\item[\hskip \labelsep{\bfseries #1\ }]\itshape}
\makeatother
\newtheorem{answer}{答}
\newcommand{\da}[1]{\begin{answer}\emph{$\!\!\!$#1}\end{answer}}
\title{高能天体物理笔记}
\author{GasinAn}
\begin{document}
    \maketitle
    \noindent Copyright \textcopyright~2025 by GasinAn

\ 

\noindent All rights reserved. No part of this book may be reproduced, 
in any form or by any means, without permission in writing from the publisher, except by a BNUer.

\ 

\noindent The author and publisher of this book have used their best efforts
in preparing this book. These efforts include the development, research, and testing of the theories,
technologies and programs to determine their effectiveness.
The author and publisher make no warranty of any kind, express or implied,
with regard to these techniques or programs contained in this book.
The author and publisher shall not be liable in any event of incidental or consequential damages
in connection with, or arising out of, the furnishing, performance, or use of these techniques or programs.

\ 

\noindent Printed in China

    \tableofcontents
\end{document}
