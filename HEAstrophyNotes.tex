% 编译方式: xelatex*2
\documentclass{ctexbook}
\usepackage{amsfonts}
\usepackage{amsmath}
\usepackage{amssymb}
\usepackage{hyperref}
\usepackage{syntonly}
\usepackage{IEEEtrantools}
%\syntaxonly
\pagestyle{plain}
\makeatletter
\newcommand{\starttoc}{
    \chapter*{\contentsname}
    \@starttoc{toc}
}
\makeatother
%\renewcommand{\tableofcontents}{\twocolumn\starttoc\onecolumn}
\hypersetup{
    colorlinks,
    linkcolor=blue,
    filecolor=pink,
    urlcolor=cyan,
    citecolor=red,
}
\def\b{\boldsymbol}
\def\d{\mathrm{d}}
\makeatletter
\def\@begintheorem#1#2{\trivlist
\item[\hskip \labelsep{\bfseries #1\ }]\itshape}
\makeatother
\newtheorem{answer}{答}
\newcommand{\da}[1]{\begin{answer}\emph{$\!\!\!$#1}\end{answer}}
\title{高能天体物理笔记}
\author{GasinAn}
\begin{document}
    \maketitle
    \noindent Copyright \textcopyright~2025 by GasinAn

\ 

\noindent All rights reserved. No part of this book may be reproduced, 
in any form or by any means, without permission in writing from the publisher, except by a BNUer.

\ 

\noindent The author and publisher of this book have used their best efforts
in preparing this book. These efforts include the development, research, and testing of the theories,
technologies and programs to determine their effectiveness.
The author and publisher make no warranty of any kind, express or implied,
with regard to these techniques or programs contained in this book.
The author and publisher shall not be liable in any event of incidental or consequential damages
in connection with, or arising out of, the furnishing, performance, or use of these techniques or programs.

\ 

\noindent Printed in China

    \tableofcontents
    \chapter{高能辐射机制简述}

\section{辐射传播相关物理量的定义}

$I_\nu(\theta,\phi)$: specific intensity (brightness, 亮度), 单位面积, 单位时间, 单位频率, 法线方向单位立体角穿过的能量,
\begin{equation}
    I_\nu(\theta,\phi)=\frac{d E}{d A\, d t\, d\nu \,d\Omega} .
\end{equation}
若 法线方向沿 $z$ 轴正方向, 且不要求法线方向穿过,
\begin{equation}
    I_\nu(\theta,\phi)=\frac{d E}{(d A\cos\theta) \,d t \,d\nu \,d\Omega} .
\end{equation}
$J_\nu$: 平均强度,
\begin{equation}
    J_\nu=\frac{1 }{4\pi}\int I_\nu(\theta,\phi)\,d\Omega .
\end{equation}
在自由空间,沿视线方向,辐射强度不变.

$F_\nu$:  monochromatic energy flux (单色能流量): 单位面积, 单位时间, 在单位频率间隔穿过的能量,
\begin{equation}
    F_\nu=\int I_\nu(\theta,\phi)\cos\theta \,d\Omega . 
\end{equation}
$\mathcal{F}$: energy fluence (能流): 单位面积穿过的能量, 
\begin{equation}
    \mathcal{F}=\int F_\nu\,dt \,d\nu .
\end{equation}

$u_\nu(\theta,\phi)$: radiative energy density 辐射能量密度,
\begin{equation}
    u_\nu(\theta,\phi)=\frac{d E}{(d A \,c\,d t )\, d\nu \,d\Omega},
\end{equation}
\begin{equation}
    u_\nu(\theta,\phi)=\frac{1}{c}I_\nu(\theta,\phi).
\end{equation}
\begin{equation}
    u_\nu=\frac{4\pi}{c}J_\nu.
\end{equation}

\section{辐射转移}

\section{辐射机制}

\subsection{黑体辐射}

\subsection{加速带电粒子的电磁辐射}

\subsubsection{电场中单电子的辐射}

\paragraph{热韧致辐射}

\paragraph{非热韧致辐射}

\subsection{磁场中单电子的辐射}

\paragraph{回旋辐射}

\paragraph{同步辐射}

\subsection{光子散射}

\subsubsection{Thomson 散射}

\subsubsection{Compton 散射}

\subsubsection{逆 Compton 散射}

    \chapter{银河系内}

\section{星际介质}

$?/xD=\theta$, $?/(1-x)D=\alpha$, $\alpha=[x/(1-x)]\theta$, $\theta_\text{sca}=\theta+\alpha=[?/x+?/(1-x)]/D=?/x(1-x)D$, $\theta=(1-x)\theta_\text{sca}$.

$\sqrt{1+\delta^2}\approx1+\frac{1}{2}\delta^2$, $xD(1+\frac{1}{2}\theta^2)+(1-x)D(1+\frac{1}{2}\alpha^2)$, $\frac{1}{2}xD\theta^2+\frac{1}{2}(1-x)D\alpha^2=\frac{1}{2}xD\theta^2+\frac{1}{2}(1-x)D[x^2/(1-x)^2]\theta^2=\frac{1}{2}xD\theta^2+\frac{1}{2}D[x^2/(1-x)]\theta^2$, $\frac{1}{2}x+\frac{1}{2}[x^2/(1-x)]=\frac{1}{2}x/(1-x)$

\section{XB}

\subsection{吸积}

吸积率: $\dot{m}$, 吸积效率: $\xi$, $L=\frac{GM}{R}\dot{m}=\xi\dot{m}c^2$.

Eddington 光度: $\frac{GMm_p}{r^2}=\frac{L\sigma_\text{T}}{4\pi r^2c}$

$L=4\pi R^2\sigma_\text{SB}T_\text{bb}^4$, $h\bar{\nu}=kT_\text{rad}$, $\frac{GMm_p}{R}=2\times\frac{3}{2}kT_\text{th}$.

    \chapter{银河系外}

\section{一般星系}

${}^{26}$Al 衰变, 1.089 MeV 谱线.

色色图: 较软能段硬度比 $\frac{M-S}{H+M+S}$-较硬能段硬度比 $\frac{H-M}{H+M+S}$.

\section{AG}

多波段, 强度高.

\subsection{AG 分类}

\subsection{AGN 统一模型}

\subsection{射电宁静 AGN}

区域: AGN 核心区域, 靠近 BH, 吸积盘内侧.

\subsection{宽线, 窄线, 反响映射}

\subsection{射电噪 AGN}

\subsubsection{视超光速}

\subsubsection{Doppler 增亮}

\section{GRB}

\end{document}
